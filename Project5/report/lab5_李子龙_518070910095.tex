\documentclass[12pt,a4paper]{article}
\usepackage[UTF8]{ctex}     %先引入ctex
\usepackage[utf8]{inputenc} %再引入inputenc
\usepackage{graphicx}
% \usepackage{lazylatex}
% \tcbuselibrary{documentation}
\usepackage{multicol}
\usepackage{tikz}
\usetikzlibrary{matrix}
\usepackage{pgfplots}
\usepgfplotslibrary{colorbrewer}
\pgfplotsset{compat=1.17}
\usepackage{xcolor}
\usepackage{listings}
\usepackage{amsmath}
\usepackage{bookmark}
\usepackage{enumerate}
\usepackage{geometry}
\graphicspath{{img/}}
% 边距
\geometry{left=2.0cm,right=2.0cm,top=2.0cm,bottom=3.0cm}
% 大题
\newenvironment{problems}{\begin{list}{}{\renewcommand{\makelabel}[1]{\textbf{##1}.\hfil}}}{\end{list}}
% 小题
\newenvironment{steps}{\begin{list}{}{\renewcommand{\makelabel}[1]{(##1)\hfil}}}{\end{list}}
% 答
\providecommand{\ans}{\textbf{答}:~}
% 解
\providecommand{\sol}{\textbf{解}.~}

% \setminted{breaklines,autogobble,frame=lines,framesep=2mm,fontsize=\scriptsize}

% listings
\definecolor{grey}{rgb}{0.8,0.8,0.8}
\definecolor{darkgreen}{rgb}{0,0.3,0}
\definecolor{darkblue}{rgb}{0,0,0.3}
\lstset{%
    numbers=left, %行号
    numberstyle=\tiny\color{grey},
    showstringspaces=false,
    showspaces=false,%
    tabsize=4,%
    frame=shadowbox,%
    basicstyle={\ttfamily\scriptsize},%
    keywordstyle=\color{blue!80!black}\bfseries,%
    commentstyle=\color{green!50!blue}\itshape,%
    stringstyle=\color{green!50!black},%
    rulesepcolor=\color{gray!20!white},
    breaklines,
    columns=flexible,
    extendedchars=false,
    %mathescape=true,
    language=c,
}

\setlength{\columnsep}{3em}

\begin{document}
\title{\normalsize \underline{计算机系统结构(A)}\\\LARGE 实验 5}
\author{李子龙 518070910095}
\date{\today}
\maketitle

\begin{problems}
    \item[一] \textbf{熟悉 SIMD intrinsics 函数}
    \begin{itemize}
        \item 4 个并行的单精度浮点数除法
        \begin{lstlisting}[identifierstyle=\color{blue}]
            __m128 _mm_div_ps (__m128 a, __m128 b)
        \end{lstlisting}
        \item 16 个并行求8 位无符号整数的最大值
        \begin{lstlisting}[identifierstyle=\color{blue}]
            __m128i _mm_max_epi16 (__m128i a, __m128i b)
        \end{lstlisting}
        \item 8 个并行的16 位带符号短整数的算术右移
        \begin{lstlisting}[identifierstyle=\color{blue}]
            __m128i _mm_bsrli_si128 (__m128i a, int imm8)
        \end{lstlisting}
    \end{itemize}
    \item[二] \textbf{阅读SIMD代码}
    仅关注函数的主要部分,执行 SIMD 操作的行用 \verb";SIMD" 标识。
    \lstset{language=[x86masm]Assembler}
	\begin{multicols}{2}
		\begin{lstlisting}
.LFB5548:
	.cfi_startproc
	endbr64
	sub	rsp, 120
	.cfi_def_cfa_offset 128
	pxor	xmm6, xmm6
	movsd	xmm1, QWORD PTR .LC2[rip]			;SIMD
	movsd	xmm10, QWORD PTR .LC0[rip]			;SIMD
	mov	rax, QWORD PTR fs:40
	mov	QWORD PTR 104[rsp], rax
	xor	eax, eax
	mov	rax, QWORD PTR .LC3[rip]
	movsd	xmm8, QWORD PTR .LC1[rip]			;SIMD
	mov	QWORD PTR 64[rsp], 0x000000000
	movsd	QWORD PTR 48[rsp], xmm1				;SIMD
	pxor	xmm9, xmm9
	lea	rsi, .LC6[rip]
	mov	edi, 1
	mov	QWORD PTR 56[rsp], rax
	movapd	xmm0, XMMWORD PTR 48[rsp]			;SIMD
	movapd	xmm3, xmm9							;SIMD
	mov	eax, 6
	movsd	QWORD PTR 32[rsp], xmm10			;SIMD
	movapd	xmm2, xmm0							;SIMD
	movsd	QWORD PTR 40[rsp], xmm8				;SIMD
	movapd	xmm7, XMMWORD PTR 32[rsp]			;SIMD
	addpd	xmm0, xmm0							;SIMD
	mov	QWORD PTR 72[rsp], 0x000000000
	mulpd	xmm2, xmm6							;SIMD
	mov	QWORD PTR 80[rsp], 0x000000000
	mulpd	xmm6, XMMWORD PTR 32[rsp]			;SIMD
	mulpd	xmm7, XMMWORD PTR .LC5[rip]			;SIMD
	mov	QWORD PTR 88[rsp], 0x000000000
	addpd	xmm7, XMMWORD PTR 64[rsp]			;SIMD
	addpd	xmm6, XMMWORD PTR 80[rsp]			;SIMD
	addpd	xmm7, xmm2							;SIMD
	addpd	xmm6, xmm0							;SIMD
	movapd	xmm2, xmm1							;SIMD
	movapd	xmm0, xmm10							;SIMD
	movapd	xmm5, xmm6							;SIMD
	movapd	xmm4, xmm7							;SIMD
	movaps	XMMWORD PTR 16[rsp], xmm6			;SIMD
	movaps	XMMWORD PTR [rsp], xmm7				;SIMD
	call	__printf_chk@PLT
	movapd	xmm6, XMMWORD PTR 16[rsp]			;SIMD
	movapd	xmm7, XMMWORD PTR [rsp]				;SIMD
	pxor	xmm9, xmm9
	mov	rax, QWORD PTR .LC1[rip]	
	movapd	xmm2, xmm9							;SIMD
	mov	edi, 1	
	lea	rsi, .LC7[rip]
	unpckhpd	xmm6, xmm6						;SIMD
	unpckhpd	xmm7, xmm7						;SIMD
	movq	xmm8, rax
	movq	xmm3, rax
	mov	rax, QWORD PTR .LC3[rip]
	movapd	xmm5, xmm6							;SIMD
	movapd	xmm4, xmm7							;SIMD	
	movapd	xmm0, xmm8							;SIMD
	movq	xmm1, rax
	mov	eax, 6
	call	__printf_chk@PLT
	mov	rax, QWORD PTR 104[rsp]
	xor	rax, QWORD PTR fs:40
	jne	.L5
	xor	eax, eax
	add	rsp, 120
	.cfi_remember_state
	.cfi_def_cfa_offset 8
	ret
    \end{lstlisting} 
	\end{multicols}
    
	\item[三] \textbf{书写 SIMD 代码}
	
	
\end{problems}
\end{document}
